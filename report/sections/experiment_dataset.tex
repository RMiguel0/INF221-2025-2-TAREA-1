Los casos de prueba considerados en esta tarea fueron generados a partir de los programas
oficiales entregados, y se dividen en dos familias: \textit{ordenamiento de arreglos unidimensionales} 
y \textit{multiplicación de matrices cuadradas}. Cada instancia se codifica mediante un nombre de archivo 
de la forma \texttt{\{n\}\_\{t\}\_\{d\}\_\{m\}.txt}, donde cada parámetro controla una característica 
del dataset y, en consecuencia, influye en el comportamiento y las mediciones de los algoritmos evaluados.

\paragraph{Ordenamiento de arreglos.} 
Los arreglos se definen por cuatro parámetros:
\begin{itemize}
    \item $n$: número de elementos, con valores en $\{10^1,10^3,10^5,10^7\}$. El tamaño del arreglo incide directamente 
    en la complejidad temporal: algoritmos cuadráticos como \texttt{InsertionSort} se vuelven impracticables 
    en $10^7$ elementos, mientras que algoritmos de orden $O(n\log n)$ (como \texttt{MergeSort} o \texttt{Quicksort}) 
    mantienen tiempos manejables.
    \item $t$: tipo de orden inicial (\textit{ascendente, descendente, aleatorio}). Este parámetro afecta de manera 
    crítica a algoritmos dependientes del orden inicial, como \texttt{InsertionSort} (muy eficiente en arreglos 
    casi ordenados, pero costoso en descendentes).
    \item $d$: dominio de los valores ($D1=\{0,\dots,9\}$ o $D7=\{0,\dots,10^7\}$). Dominios pequeños generan 
    muchos valores repetidos, lo cual impacta la estabilidad y las comparaciones internas del algoritmo.
    \item $m$: muestra aleatoria (\textit{a,b,c}), que permite obtener variabilidad en los casos aún con los mismos 
    parámetros, asegurando que los resultados no dependan de una única instancia no representativa.
\end{itemize}

\paragraph{Multiplicación de matrices.} 
En este caso, cada par de archivos define dos matrices cuadradas $M_1, M_2$ cuyos parámetros son:
\begin{itemize}
    \item $n$: dimensión de la matriz, con valores en $\{2^4,2^6,2^8,2^{10}\}$. El tamaño domina tanto el tiempo 
    como la memoria: mientras \texttt{Naive} ejecuta $O(n^3)$ operaciones, \texttt{Strassen} reduce a 
    $O(n^{\log_2 7}) \approx O(n^{2.81})$, pero con sobrecosto en memoria.
    \item $t$: tipo de matriz (\textit{dispersa, diagonal, densa}). Matrices dispersas o diagonales reducen 
    el número de operaciones efectivas, lo que se traduce en tiempos más bajos que en el caso denso, donde 
    prácticamente todos los elementos participan en los cálculos.
    \item $d$: dominio de valores ($D0=\{0,1\}$ o $D10=\{0,\dots,9\}$). Un dominio más amplio genera más diversidad 
    numérica y menor probabilidad de estructuras degeneradas (ej. filas nulas), lo que puede impactar tanto 
    en la dificultad del cálculo como en la validez de comparaciones entre algoritmos.
    \item $m$: muestra aleatoria (\textit{a,b,c}), que permite disponer de tres variantes de cada configuración, 
    evitando sesgos por un caso aislado y favoreciendo un análisis estadístico más robusto.
\end{itemize}

En conjunto, estos parámetros aseguran una batería de pruebas diversa que cubre desde instancias pequeñas y 
estructuradas hasta entradas de gran escala y aleatorias, lo cual permite evaluar no sólo la complejidad teórica 
de los algoritmos, sino también su rendimiento práctico bajo diferentes condiciones.
