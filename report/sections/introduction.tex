\begin{mdframed}
    \textbf{La extensión máxima para esta sección es de 2 páginas.}
\end{mdframed}

La introducción de este tipo de informes o reportes, tiene como objetivo principal \textbf{contextualizar el problema que se va a analizar}, proporcionando al lector la información necesaria para entender la relevancia del mismo. 

Es fundamental que en esta sección se presenten los antecedentes del problema, destacando investigaciones previas o principios teóricos que sirvan como base para los análisis posteriores. Además, deben explicarse los objetivos del informe, que pueden incluir la evaluación de un algoritmo, la comparación de métodos o la validación de resultados experimentales.

Aunque la estructura y el enfoque siguen principios de trabajos académicos, se debe recordar que estos informes no son publicaciones científicas formales, sino trabajos de pregrado. Por lo tanto, se busca un enfoque claro y directo, que permita al lector comprender la naturaleza del problema y los objetivos del análisis, sin entrar en detalles excesivos. 


Introduction Checklist de \citetitle{GoodScientificPaper} \cite{GoodScientificPaper}, adaptada a nuestro contexto:

\begin{itemize}
\item Indique el \textbf{campo del trabajo} (Análisis y Diseño de algoritmos en Ciencias de la Computación), por qué este campo es importante y qué se ha hecho ya en este área, con las \textbf{citas} adecuadas de la literatura académica o fuentes relevantes.
\item Identifique una \textbf{brecha} en el conocimiento, un desafío práctico, o plantee una \textbf{pregunta} relacionada con la eficiencia, complejidad o aplicabilidad de un algoritmo particular.
\item Resuma el propósito del informe e introduzca el análisis o experimento, dejando claro qué se está investigando o comparando, e indique \textbf{qué es novedoso} o por qué es significativo en el contexto de un curso de pregrado.
\item Evite; repetir el resumen; proporcionar información innecesaria o fuera del alcance de la materia (limítese al análisis de algoritmos o conceptos de complejidad); exagerar la importancia del trabajo (recuerde que se trata de un informe de pregrado); afirmar novedad sin una comparación adecuada con lo enseñado en clase o la bibliografía recomendada.
\end{itemize}



\begin{mdframed}
Recuerde que este es su trabajo, y sólo usted puede expresar con precisión lo que ha aprendido y quiere transmitir. Si lo hace bien, su introducción será más significativa y valiosa que cualquier texto automatizado. ¡Confíe en sus habilidades, y verá que puede hacer un mejor trabajo que cualquier herramienta que automatiza la generación de texto!
\end{mdframed}

