Este reporte analiza experimentalmente distintos algoritmos de ordenamiento y de multiplicación de matrices bajo escenarios que simulan condiciones de estrés computacional. 
El objetivo fue contrastar su comportamiento práctico con las complejidades teóricas asociadas. 
A partir de los resultados obtenidos y de las gráficas generadas, se observa que la elección del algoritmo más adecuado depende en gran medida del caso de uso específico. 
Si bien las tendencias teóricas se cumplen en términos generales, las constantes ocultas y el manejo de memoria tienen un impacto decisivo en el rendimiento real, 
lo que puede volver poco convenientes a ciertos algoritmos en instancias de tamaño moderado.