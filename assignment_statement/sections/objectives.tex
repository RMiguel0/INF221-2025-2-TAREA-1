El objetivo de esta tarea \tnum~ es introducirnos
en el Análisis experimental de Algoritmos. Para ello,
se propone realizar un estudio experimental de 4 algoritmos
de ordenamiento y 2 de multiplicación de matrices. \\

Para cada uno de los dos problemas, el de ordenar un arreglo unidimensional de números enteros, y el de multiplicar dos matrices cuadradas, existen múltiples algoritmos que los resuelven, y la mayoría, si no todos, tienen implementaciones de fácil acceso (ya se encuentran en bibliotecas de lenguajes de programación, o son fáciles de encontrar en internet). Sus complejidades teóricas son conocidas, ya que estás son una característica fundamental a la hora de diseñar algoritmos y, en la mayoría de los casos, la motivación para diseñarlos. \\

El desafío en esta tarea no está en diseñar los algoritmos, ya que como se mencionó anteriormente, estos ya están diseñados y muchas veces implementados, ni tampoco en calcular o demostrar su complejidad teórica: el desafío está en estudiar su comportamiento en la práctica y relacionar este comportamiento con la complejidad teórica. \\





