
  \begin{enumerate}[(1)]
  \item
    La tarea se realizará individualmente
    (esto es grupos de una persona),
    sin excepciones.
  \item
    La entrega debe realizarse vía \url{http://aula.usm.cl}, entregando la url del repositorio privado de GitHub donde se encuentra el código fuente de su tarea. Este repositorio debe ser un FORK del repositorio 
    \begin{center}

      \url{https://github.com/pabloealvarez/INF221-2025-1-TAREA-1/tree/master/code}
          
  \end{center}

  \begin{itemize}
      
  \item El repositorio de github \textbf{DEBE SER PRIVADO}, ya que de lo contrario cualquier persona podrá acceder a su código y cometer plagio, siendo usted responsable de ello.
  \item  \textbf{DEBERÁ DAR ACCESO A LOS AYUDANTES DE SU RESPECTIVO CAMPUS}, antes de la fecha de entrega. Para ello, se informará antes de la fecha de entrega el nombre de usuario de los ayudantes de su respectivo campus y deberá agregar a los ayudantes como colaboradores del repositorio que contiene su entrega. 
  \end{itemize}

  \item Si se utiliza algún código, idea, o contenido extraído de otra fuente, este \textbf{debe} ser citado en el lugar exacto donde se utilice. 
  \item
  Asegúrese que todas sus entregas tengan sus datos completos:
  número de la tarea, ramo, semestre, nombre y rol.
  Puede incluirlas como comentarios en sus fuentes \LaTeX{}
  (en \TeX{} comentarios son desde \% hasta el final de la línea)
  o en posibles programas.
  Anótese como autor de los textos.

  \item
    Si usa material adicional al discutido en clases,
    detállelo.
    Agregue información suficiente para ubicar ese material
    (en caso de no tratarse de discusiones con compañeros de curso
     u otras personas).
    \item No modifique \texttt{preamble.tex}, \texttt{report.tex}, \texttt{rules.tex}, estructura de directorios, nombres de archivos, configuración del documento, etc. Sólo agregue texto, imágenes, tablas, código, etc. En el códigos fuente de su informe/reporte, no agregue paquetes, ni archivos .tex.

  \item

    La fecha límite de entrega es el día \tcm{\deadline}.

    \begin{center}
        \Large{
          \textbf{NO SE ACEPTARÁN TAREAS FUERA DE PLAZO}.
        }
        \normalsize
    \end{center}
     
    
  \item
    Nos reservamos el derecho de llamar a interrogación
    sobre algunas de las tareas entregadas.
    En tal caso,
    la nota de la tarea será la obtenida en la interrogación.
    \begin{center}
      \Large{
        \textbf{NO PRESENTARSE A UN LLAMADO A INTERROGACIÓN SIN JUSTIFICACIÓN PREVIA SIGNIFICA AUTOMÁTICAMENTE NOTA 0.}
      }
    \end{center}
    
  \end{enumerate}
