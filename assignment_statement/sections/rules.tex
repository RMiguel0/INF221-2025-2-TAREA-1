
  \begin{itemize}
  \item
    La tarea se realizará \tcm{individualmente}
    (esto es grupos de una persona),
    sin excepciones.
  \item
    La entrega debe realizarse vía \url{http://aula.usm.cl}
    en un \tcm{tarball} en el área designada al efecto,
    en el formato \tcm{\texttt{tarea-\tnum-{rol}.tar.gz}}
    (\texttt{rol} con dígito verificador y sin guión).

    Dicho \tcm{tarball} debe contener la misma estructura del repositorio: mismos directorios y archivos, pero con los contenidos de los archivos que usted ha modificado para su entrega.

  \item Si se utiliza algún código, idea, o contenido extraído de otra fuente, este \textbf{debe} ser citado en el lugar exacto donde se utilice, en lugar de mencionarlo al final del informe. 
  \item
    Asegúrese que todas sus entregas tengan sus datos completos:
    número de la tarea, ramo, semestre, nombre y rol.
    Puede incluirlas como comentarios en sus fuentes \LaTeX{}
    (en \TeX{} comentarios son desde \% hasta el final de la línea)
    o en posibles programas.
    Anótese como autor de los textos.
 
  \item
    Si usa material adicional al discutido en clases,
    detállelo.
    Agregue información suficiente para ubicar ese material
    (en caso de no tratarse de discusiones con compañeros de curso
     u otras personas).
    \item No modifique \texttt{preamble.tex}, \texttt{tarea\_main.tex}, \texttt{condiciones.tex}, estructura de directorios, nombres de archivos, configuración del documento, etc. Sólo agregue texto, imágenes, tablas, código, etc. En el códigos funte de su informe, no agregue paquetes, ni archivos .tex (a excepción de que agregue archivos en \texttt{/tikz}, donde puede agregar archivos .tex con las fuentes de gráficos en \texttt{TikZ}).

  \item

    La fecha límite de entrega es el día \tcm{\deadline}.

    \begin{center}
        \Large{
          \textbf{NO SE ACEPTARÁN TAREAS FUERA DE PLAZO}.
        }
        \normalsize
    \end{center}
     
    
  \item
    Nos reservamos el derecho de llamar a interrogación
    sobre algunas de las tareas entregadas.
    En tal caso,
    la nota de la tarea será la obtenida en la interrogación.
    \begin{center}
      \Large{
        \textbf{NO PRESENTARSE A UN LLAMADO A INTERROGACIÓN SIN JUSTIFICACIÓN PREVIA SIGNIFICA AUTOMÁTICAMENTE NOTA 0.}
      }
    \end{center}
    
  \end{itemize}
