\subsubsection{Ordenamiento de un arreglo unidimensional de números enteros}

\textbf{Entrada:}
\begin{itemize}
    \item Leer un arreglo unidimensional $A$ desde el archivo \texttt{\{n\}\_\{t\}\_\{d\}\_\{m\}.txt}.
    \begin{itemize}
        \item $n$ hace referencia a la cantidad de elementos (o largo del arreglo) y pertenece al conjunto $\mathcal{N} = \{10^1, 10^3, 10^5, 10^{7}\}$.
        \item $t$ hace referencia al tipo de matriz, y pertenece al conjunto $\mathcal{T} = \{ \text{ascendente}, \text{descendente}, \text{aleatorio} \}$.
        \item $d$ hace referencia al conjunto dominio de cada elemento del arreglo. $d = \{D1, D7\}$, donde $D7$ implica que el dominio es $\{0,1,2,3,4,5,6,7,8,9\}$ y $D7 $ que el dominio es $ \{0,1,2,3,...,10^7\}$.
        \item $m$ hace referencia a la muestra aleatoria (o caso de prueba) y pertenece al conjunto $\mathcal{M} = \{a,b,c\}$.
    \end{itemize}
\end{itemize}

\textbf{Salida:}
\begin{itemize}
    \item Escribir la matriz resultante $M_1 \times M_2$ en el archivo \texttt{\{n\}\_\{t\}\_\{d\}\_\{m\}\_out.txt}.
\end{itemize}

\subsubsection{Multiplicación de matrices cuadradas de números eneteros}

\textbf{Entrada:}
\begin{itemize}
    \item Leer dos matrices cuadradas de entrada $M_1$ y $M_2$ desde los archivos \texttt{\{n\}\_\{t\}\_\{d\}\_\{m\}\_\_1.txt} y \\ 
    \texttt{\{n\}\_\{t\}\_\{d\}\_\{m\}\_2.txt}, respectivamente.
    \begin{itemize}
        \item $n$ hace referencia a la dimensión de la matriz ($n$ filas y $n$ columnas) y pertenece al conjunto $\mathcal{N} = \{2^4, 2^6, 2^8, 2^{10}\}$.
        \item $t$ hace referencia al tipo de matriz, y pertenece al conjunto $\mathcal{T} = \{ \text{dispersa}, \text{diagonal}, \text{densa} \}$.
        \item $d$ hace referencia al dominio de cada coeficiente de la matriz $d = \{D0, D10\}$, donde $D0$ implica que el dominio es $\{0,1\}$ y $D10$ que el dominio es $\{0,1,2,3,4,5,6,7,8,9\}$. 
        \item $m$ hace referencia a la muestra aleatoria (o caso de prueba) y pertenece al conjunto $\mathcal{M} = \{a,b,c\}$.
    \end{itemize}
\end{itemize}

\textbf{Salida:}
\begin{itemize}
    \item Escribir la matriz resultante $M_1 \times M_2$ en el archivo \texttt{\{n\}\_\{t\}\_\{d\}\_\{m\}\_out.txt}.
\end{itemize}


