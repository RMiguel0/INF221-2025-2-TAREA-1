En esta sección se describen los pasos a seguir para completar la tarea, la cual consiste en desarrollar código en C++ <<\nameref{subsec:implementations}>> y en realizar un informe <<\nameref{subsec:report}>>. Se espera que cada uno de los pasos se realice de manera ordenada y siguiendo las instrucciones dadas.

\begin{mdframed}
    Abra este documento en algún lector de pdf que permita hipervínculos, ya que en este documento el texto en color \textcolor{blue}{azul} suele indicar un hipervínculo.
\end{mdframed}

\begin{enumerate}[(1)]
    \item 
    En caso de cualquier duda, contactarse directamente con \textbf{Pablo Álvarez}. Para esto pueden hacerlo por mensaje directo en discord (\url{pabloealvarez}) o correo electrónico (\url{pablo.alvarezs@sansano.usm.cl}). En caso de cualquier de modificaciones, todas se informarán tanto por aula como por discord.
    \item Todo lo necesario para realizar la tarea se encuentra en la \textbf{rama master} del repositorio de github:
\begin{center}

    \url{https://github.com/pabloealvarez/INF221-2025-1-TAREA-1/tree/master/code}
        
\end{center}
\item En el repositorio del punto (2) pueden existir otras ramas, pero master siempre será donde se encuentra la información oficial.
\end{enumerate}

\subsection{Implementaciones} \label{subsec:implementations}




\begin{enumerate}[(1)]
    \item Implementar cada uno los algoritmos en C++:
    \begin{itemize}
        \item Para el problema de ordenamiento de un arreglo unidimensional de números enteros, se deben implementar en C++, en los archivos correspondientes, los algoritmos de ordenamiento \textsc{Selection Sort}, \textsc{Merge Sort}, \textsc{Quick Sort} y el algoritmo \textsc{Sort} de la librería estándar de C++. 
    \begin{itemize}
        \item \href{https://github.com/pabloealvarez/INF221-2025-1-TAREA-1/blob/master/code/sorting/algorithms/selectionsort.cpp}{code/sorting/algorithms/selectionsort.cpp}
        \item \href{https://github.com/pabloealvarez/INF221-2025-1-TAREA-1/blob/master/code/sorting/algorithms/mergesort.cpp}{code/sorting/algorithms/mergesort.cpp}
        \item \href{https://github.com/pabloealvarez/INF221-2025-1-TAREA-1/blob/master/code/sorting/algorithms/quicksort.cpp}{code/sorting/algorithms/quicksort.cpp}
        \item \href{https://github.com/pabloealvarez/INF221-2025-1-TAREA-1/blob/master/code/sorting/algorithms/sort.cpp}{code/sorting/algorithms/sort.cpp}
    \end{itemize}
    El algoritmo \textsc{Sort} ya se encuentra implementado en el repositorio con el fin de que sea incluido en sus mediciones del punto (2) de la subsección <<\nameref{subsec:implementations}>>.
    
        \item Para el problema de multiplicación de matrices cuadradas de números enteros, se deben implementar en C++, en los archivos correspondientes, los algoritmos de multiplicación \textsc{Naive} y \textsc{Strassen}.
    \begin{itemize}
        \item \href{https://github.com/pabloealvarez/INF221-2025-1-TAREA-1/blob/master/code/matrix_multiplication/algorithms/naive.cpp}{code/matrix\_multiplication/algorithms/naive.cpp}
        \item \href{https://github.com/pabloealvarez/INF221-2025-1-TAREA-1/blob/master/code/matrix_multiplication/algorithms/strassen.cpp}{code/matrix\_multiplication/algorithms/strassen.cpp}
    \end{itemize}
    \end{itemize}

    \item Implementar el programa que realizará las mediciones de tiempo y memoria en C++ (programas principales) y su respectivo \texttt{makefile}, que ejecutará los algoritmos y generará los archivos de salida en cada uno de los directorios \href{https://github.com/pabloealvarez/INF221-2025-1-TAREA-1/blob/master/code/sorting/data/measurements/}{measurements sorting} y \href{https://github.com/pabloealvarez/INF221-2025-1-TAREA-1/blob/master/code/sortmatrix+multiplication/data/measurements/}{measurements matrix multiplication} con los resultados de cada uno de los algoritmos.
    \begin{itemize}
        \item \href{https://github.com/pabloealvarez/INF221-2025-1-TAREA-1/blob/master/code/sorting/sorting.cpp}{code/sorting/sorting.cpp}
        \item \href{https://github.com/pabloealvarez/INF221-2025-1-TAREA-1/blob/master/code/matrix_multiplication/matrix_multiplication.cpp}{code/matrix\_multiplication/matrix\_multiplication.cpp}
    \end{itemize}
    
    \item Implementar el programa que generará los gráficos en PYTHON y que se encargará de leer los archivos generados por los programas principales guardados en  \href{https://github.com/pabloealvarez/INF221-2025-1-TAREA-1/blob/master/code/sorting/data/measurements/}{measurements sorting} y \href{https://github.com/pabloealvarez/INF221-2025-1-TAREA-1/blob/master/code/sortmatrix+multiplication/data/measurements/}{measurements matrix multiplication}, para luego graficar los resultados obtenidos y guardarlos en formato PNG en  \href{https://github.com/pabloealvarez/INF221-2025-1-TAREA-1/blob/master/code/sorting/data/plots/}{plots sorting} y \href{https://github.com/pabloealvarez/INF221-2025-1-TAREA-1/blob/master/code/sortmatrix+multiplication/data/plots/}{plots matrix multiplication}.
    \begin{itemize}
        \item \href{https://github.com/pabloealvarez/INF221-2025-1-TAREA-1/blob/master/code/sorting/scripts/array_generator.py}{code/sorting/algorithms/scripts/plot\_generator.py}
        \item \href{https://github.com/pabloealvarez/INF221-2025-1-TAREA-1/blob/master/code/matrix_multiplication/scripts/matrix_generator.py}{code/matrix\_multiplication/scripts/plot\_generator.py}
    \end{itemize}
    \item Documentar Cada uno de los pasos anteriores
    \begin{itemize}
        \item Completar el archivo \texttt{README.md} del directorio \texttt{code}
        \item Documentar en cada uno de sus programas, al inicio de cada archivo, fuentes de información, referencias y/o bibliografía utilizada para la implementación de cada uno de los algoritmos.
    \end{itemize}
\end{enumerate}

\subsection{Informe} \label{subsec:report}

Luego de realizar las implementaciones y experimentos, se debe generar un informe en \LaTeX\ que contenga los resultados obtenidos y una discusión sobre ellos. En el siguiente repositorio podrá encontrar el \href{https://github.com/pabloealvarez/INF221-2025-1-TAREA-1/tree/master/report}{Template} que \textbf{deben utilizar}, en esta entrega:

\begin{mdframed}

\begin{center}
    
    \url{https://github.com/pabloealvarez/INF221-2025-1-TAREA-1/tree/master/report}
        
\end{center}
\end{mdframed}

\begin{itemize}
    \item No se debe modificar la estructura del informe.
    \item Las indicaciones se encuentran en el archivo \texttt{README.md} del repositorio y en la plantilla de \LaTeX. 
\end{itemize}

